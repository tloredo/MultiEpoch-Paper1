%--------------------------------------------------------------------------------
\iffalse
\subsection{Mixed Detections of Object and Noise}
\label{sec:mix}
\noindent
The faintest sources rarely pop above the threshold, which makes them apparently more distiguishable from the noise, as seen previously in Figure~\ref{fig:bf}. This is a real effect due to the fact that the noise peaks tend to be higher flux; see details in the Appendix. In a real survey, candidate associations of faint sources are expected to be contaminated  by noise peaks. To deal with this, we can  extend our model to include a label parameter $v_i$ for each detection, which takes the value of, say, 1 if the detection is from the object, and 0 if it is a ghost.
%
Formally this is similar to sorting the measurements into correct and incorrect classes for robust measurements, as discussed by \citet{press}.
%
The likelihood is a product of the two possibilities, which is a function of the set of labels $\apjvec{v}$ and the flux.
%
\begin{equation}
L(\apjvec{v},f)= (1\!-\!P_f)^{k\!-\!n} \prod_{v_i=1}\!G(f_i;f,\sigma^2) \prod_{v_i=0}\!(1\!-\!P_f)\,N(f_i;\sigma^2)
\end{equation}
%
If $\beta$ represents the probability that a detection belongs to the object, the conditional probability of $\apjvec{v}$ is
%
\begin{equation}
\pi(\apjvec{v}|\beta) = \prod_{v_i=1}\!\beta \prod_{v_i=0}\!(1\!-\!\beta)
\end{equation}
%
Not knowing the value of $\beta$ we can assume its prior to be flat on the [0,1] interval to derive the posterior probability, which can be marginalized to find the best labels. Also we can perform the full integration over all parameters to use for hypothesis testing.
%
\begin{eqnarray}
L_{\text{mix}} & = & \int_0^1\!\!d\beta \int\!\!df\,\pi(f)\ (1\!-\!P_f)^{k\!-\!n} \ \times\\
 & \times & \sum_{\apjvec{v}}\prod_{v_i=1}\!\beta G(f_i;f,\sigma^2) \prod_{v_i=0}\!(1\!-\!\beta)(1\!-\!P_f)\,N(f_i;\sigma^2) \nonumber \\
 & = & \int_0^1\!\!d\beta \int\!\!df\ \pi(f)\ (1\!-\!P_f)^{k\!-\!n}  \ \times \\
 & \times & \prod_i \Big[\beta G(f_i;f,\sigma^2) + (1\!-\!\beta)(1\!-\!P_f)\,N(f_i;\sigma^2) \Big] \nonumber
\end{eqnarray}
%
In confused fields this can be further extended to include the possiblity of multiple objects \citep{loredo}.




\iffalse
\begin{figure}
\epsscale{1.2}
\plotone{fig/f9.png}
\caption{The Bayes factor increases for the simulate noise peaks as we increase their surface density. These simulations illustrate that effect for 1$\times$, 2$\times$, ..., and 5$\times$ the original density used in Figure~\ref{fig:bf2}. The highest density of about 3/$\square\arcsec$ corresponds roughly to 1 peak in 9 pixels for images with 0.5\arcsec pixels, which can be considered the absolute limit.}
\label{fig:bfx}
\end{figure}
\fi

